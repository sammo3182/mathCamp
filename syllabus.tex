\documentclass[11pt, a4paper]{article}
%\usepackage{geometry}
\usepackage[inner=2.5cm,outer=2.5cm,top=2.5cm,bottom=2.5cm]{geometry}
\pagestyle{empty}
\usepackage{graphicx}
\usepackage{fancyhdr, lastpage, bbding, pmboxdraw}
\usepackage[usenames,dvipsnames]{color}
\definecolor{darkblue}{rgb}{0,0,.6}
\definecolor{darkred}{rgb}{.7,0,0}
\definecolor{darkgreen}{rgb}{0,.6,0}
\definecolor{red}{rgb}{.98,0,0}
\usepackage[colorlinks,pagebackref,pdfusetitle,urlcolor=darkblue,citecolor=darkblue,linkcolor=darkred,bookmarksnumbered,plainpages=false]{hyperref}
\renewcommand{\thefootnote}{\fnsymbol{footnote}}

%\pagestyle{fancyplain}
%\fancyhf{}
%\lhead{ \fancyplain{}{Modeling \&  Evaluation} }
%\chead{ \fancyplain{}{} }
%\rhead{ \fancyplain{}{September 26, 2014} }
%\rfoot{\fancyplain{}{page \thepage\ of \pageref{LastPage}}}
%\fancyfoot[RO, LE] {page \thepage\ of \pageref{LastPage} }
%\thispagestyle{plain}

%%%%%%%%%%%% LISTING %%%
\usepackage{listings}
\usepackage{caption}
\DeclareCaptionFont{white}{\color{white}}
\DeclareCaptionFormat{listing}{\colorbox{gray}{\parbox{\textwidth}{#1#2#3}}}
\captionsetup[lstlisting]{format=listing,labelfont=white,textfont=white}
\usepackage{verbatim} % used to display code
\usepackage{fancyvrb}
\usepackage{acronym}
\usepackage{amsthm}
\VerbatimFootnotes % Required, otherwise verbatim does not work in footnotes!



\definecolor{OliveGreen}{cmyk}{0.64,0,0.95,0.40}
\definecolor{CadetBlue}{cmyk}{0.62,0.57,0.23,0}
\definecolor{lightlightgray}{gray}{0.93}



\lstset{
%language=bash,                          % Code langugage
basicstyle=\ttfamily,                   % Code font, Examples: \footnotesize, \ttfamily
keywordstyle=\color{OliveGreen},        % Keywords font ('*' = uppercase)
commentstyle=\color{gray},              % Comments font
numbers=left,                           % Line nums position
numberstyle=\tiny,                      % Line-numbers fonts
stepnumber=1,                           % Step between two line-numbers
numbersep=5pt,                          % How far are line-numbers from code
backgroundcolor=\color{lightlightgray}, % Choose background color
frame=none,                             % A frame around the code
tabsize=2,                              % Default tab size
captionpos=t,                           % Caption-position = bottom
breaklines=true,                        % Automatic line breaking?
breakatwhitespace=false,                % Automatic breaks only at whitespace?
showspaces=false,                       % Dont make spaces visible
showtabs=false,                         % Dont make tabls visible
columns=flexible,                       % Column format
morekeywords={__global__, __device__},  % CUDA specific keywords
}

%%%%%%%%%%%%%%%%%%%%%%%%%%%%%%%%%%%%
\begin{document}
\begin{center}
{\Large \textsc{Math Camp}}
\end{center}
\begin{center}
Summer 2017
\end{center}
%\date{September 26, 2014}

\begin{center}
\rule{6in}{0.4pt}
\begin{minipage}[t]{.75\textwidth}
\begin{tabular}{llcccll}
\textbf{Instructor:} & Yue Hu & & &  &  & Jielu Yao  \\
Email: &  \href{mailto:yue-hu-1@uiowa.edu}{yue-hu-1@uiowa.edu} & & & &  & \href{mailto:jielu-yao@uiowa.edu}{jielu-yao@uiowa.edu} 
\end{tabular}
\end{minipage}
\rule{6in}{0.4pt}
\end{center}
\vspace{.5cm}
\setlength{\unitlength}{1in}
\renewcommand{\arraystretch}{2}

\section{Course Description}
The five sessions of Math Camp will cover the key concepts and theories in basic algebra, linear algebra, and calculus, underlying most work in quantitative political research. 
Each session consists of a 120-minute lecture including exercises. 
You are free, but not required, to read the textbook or do some exercises on your own before the session.

You do not need to know anything about calculus or linear algebra to attend the session. 
Nor do you need to be anxious. 
An ultimate goal of these sessions is to help you overcome the fear of learning and applying math. 
So, don't worry, we will start from scratch.
If you believe that you already have sufficient mathematic knowledge, there is still a value to attend the lectures. 
You will learn how the knowledge and skills are applied in the political methodology and how they will help study the substantive political topics you are interested.
Moreover, it is also a good time for you to know your cohort and make friends.

\section{Textbook and Resources}
Moore, Will and David Siegel. 2013. {\it A Mathematics Course for Political and Social Research}. Princeton University Press.

This book can be accessed online through the UI library website. You don't have to buy it. But it is a good mathematical reference for political science students. There is a YouTube channel consisting of videos on every topic covered in the book.\\
\href{https://www.youtube.com/channel/UCrA2SLUKnV6yjdgIfDwFeGg/playlists}{https://www.youtube.com/channel/UCrA2SLUKnV6yjdgIfDwFeGg/playlists} 

\begin{itemize}
\item[o] Other Suggested Readings \\
Hagel, Timothy. 1995. {\it Basic Math for Social Scientists: Concepts. Sage Publications.} \\
Gill, Jeff. 2006. {\it Essential Mathematics for Political and Social Research. Cambridge University Press.} 

\item[o] Video Resources \\
Khan Academy \\
\href{https://www.khanacademy.org/}{https://www.khanacademy.org/} \\
MIT Video Courses \\
\href{http://ocw.mit.edu/courses/audio-video-courses/}{http://ocw.mit.edu/courses/audio-video-courses/}
\end{itemize}

\section{Schedule} 
\begin{itemize}
\item[(1)]August 14, Monday \underline{Building blocks} 
\item[o] 10:00am - noon: Notations, operators, and sets
\item[o] 1:30pm – 3:30pm: Exponents, logarithm, and functions
\item[o] {\it Reading: Moore \& Siegel, Chapter 1, 3} 
\end{itemize}

\begin{itemize}
\item[(2)]August 15, Tuesday \underline{Linear algebra and Calculus}
\item[o] 10:00am - noon: Vectors and matrices
\item[o] 1:30pm – 3:30pm: Differentiation
\item[o] {\it Reading: Moore \& Siegel, Chapter 5, 6 \& 12} 
\end{itemize}

\begin{itemize} 
\item[(3)]August 18, Friday \underline{More Calculus} 
\item[o] 10:00am - noon: Integration
\item[o] {\it Reading: Moore \& Siegel, Chapter 7} 
\end{itemize}

%%%%%% END 
\end{document} 